%%%%%%%%%%%%%%%%%%%%%%%%%%%%%%%%%%%%%%%% Pakete
\documentclass[10pt,a4paper]{article}

%Schrift-Packages_7
\usepackage{textcomp}
\usepackage{fixltx2e}
\usepackage[ngerman]{babel}		%% neue deutsche Rechtschreibung
\usepackage[utf8x]{inputenc}    %% fügt Umlaute hinzu
\PrerenderUnicode{ä}            %% für Umlaute auf der Titelseite 
\PrerenderUnicode{ö}
\PrerenderUnicode{ü}
\PrerenderUnicode{ß}
\usepackage[T1]{fontenc}        %% for selecting font encodings
\usepackage{lmodern}            %% besseres Schriftbild für PDFs
\usepackage{ucs}                %% extended UTF-8 input encoding support

\usepackage{amssymb}            %% extended math symbol collection
\usepackage{amsmath}            %% miscellaneous enhancements for formulas   
\usepackage{graphicx}           %% für das Einfügen von Bildern
\usepackage{pdfpages}           %% für das Einbinden von PDFs



\usepackage{geometry}           %% Gestaltung einer Seite
\geometry{a4paper}              %% Voreinstellung für A4-Seiten

\usepackage{booktabs}           %% enhances the quality of tables
\usepackage{float}              %% Improves the interface for defining floating objects such as figures and tables
\usepackage[absolute]{textpos}  %% facilitate placement of boxes at absolute positions
\usepackage{ulem}               %% Unterstreichen von Absätzen

\usepackage[skins]{tcolorbox}   %% environment for coloured and framed text boxes

\usepackage{subfigure}          %% manipulation and reference of small or 'sub' figures and tables 

\usepackage{titling}            %% control over the typesetting of the \maketitle command and \thanks commands

\usepackage{fancyhdr}           %% control of page headers and footers

\definecolor{fh_green}{RGB}{121,167,24}     %% FH-Farben werden hier definiert
\definecolor{fh_grey}{RGB}{95,100,103}
\definecolor{fh_blue}{RGB}{10,79,138} 

\pagestyle{fancy}               
\fancyhead[R]{\leftmark}        %% Voreinstellung der Kopfzeile
%%%%%%%%%%%%%%%%%%%%%%%%%%%%%%%%%%%
%%%%%%%%%%%%%%%%%%%%%%%%%%%%%%%%%%%
%%%%%%%%%%%%%%%%%%%%%%%%%%%%%%%%%%% 

%%% Hier Kapitel, Titel, AutorInnen, Studiengang, Lehrveranstaltung, BetreuerIn und Erstellungsdatum angeben: %%%

\newcommand{\thema}{Fehlerfortpflanzung Berechnung}
\title{Labor Übung Thermodynamik: Fehlerfortpflanzung Wärme Kapazität}
\author{John Bryan Valle, Paulus Summer}
\newcommand{\studiengang}{Elektronik}
\newcommand{\lehrveranstaltung}{PHYLAB}
\newcommand{\betreuerin}{Brigitte Waldmann MEd.}
\newcommand{\erstellungsdatum}{07.10.2023}


%%%%%%%%%%%%%%%%%%%%%%%%%%%%%%%%%%%
%%%%%%%%%%%%%%%%%%%%%%%%%%%%%%%%%%%
%%%%%%%%%%%%%%%%%%%%%%%%%%%%%%%%%%% Doument (Anfang)
\begin{document}
%%%%%%%%%%%%%%%%%%%%%%%%%%%%%%%%%%%
%%%%%%%%%%%%%%%%%%%%%%%%%%%%%%%%%%%
%%%%%%%%%%%%%%%%%%%%%%%%%%%%%%%%%%% Titelseite (Anfang)
\begin{titlepage}

\begin{textblock}{3}(-0.4,0)
\includegraphics{bilder/Deckblatt.pdf}
\end{textblock}
%Titel
\mbox{}

\begin{textblock}{8}(2,3)

\noindent \Huge \textbf{\thema}

\noindent \Huge  \thetitle
\end{textblock}

%Author & Datum
\begin{textblock}{12.5}(1.5,11.5)

\Large Erstellt von: \hfill \theauthor

\Large Studiengang: \hfill \studiengang

\Large Lehrveranstaltung: \hfill \lehrveranstaltung

\Large BetreuerIn: \hfill \betreuerin

\Large Wien, am \erstellungsdatum 
\end{textblock}
\end{titlepage}
%%%%%%%%%%%%%%%%%%%%%%%%%%%%%%%%%%%
%%%%%%%%%%%%%%%%%%%%%%%%%%%%%%%%%%%
%%%%%%%%%%%%%%%%%%%%%%%%%%%%%%%%%%% Titelseite (Ende)
\newpage
%%%%%%%%%%%%%%%%%%%%%%%%%%%%%%%%%%% Inhaltsverzeichnis (Anfang)
\thispagestyle{plain}
\tableofcontents
%%%%ö%%%%%%%%%%%%%%%%%%%%%%%%%%%%%% Inhaltsverzeichnis (Ende)
\newpage
%%%%%%%%%%%%%%%%%%%%%%%%%%%%%%%%%%% Inhalt (Anfang)

\section{Fehlerfortpflanzung: Herleitung $c_w$ auf Basis der Beziehung zwischen Potentiellen Energie $U$ und der Wärme Energie $Q$}
\begin{equation}
    Q = m c \Delta T
\end{equation}

\begin{equation}
    c_w = \frac{Q}{m_w \Delta T}
\end{equation}

\begin{equation}
    c_w = \frac{P t}{m_w \Delta T}
\end{equation}

\begin{equation}
    c_w = \frac{\frac{U^2}{R}}{m_w \frac{\Delta T}{t}}
\end{equation}

\begin{equation}
    c_w = \frac{U^2}{R m_w \frac{\Delta T}{t}}
\end{equation}

\subsection{Partielle Ableitung nach U}

\begin{equation}
    \frac{\partial c_w}{\partial U} = \frac{2 U}{R m_w \frac{\Delta T}{\Delta t}}
\end{equation}

\subsection{Partielle Ableitung nach $\frac{\Delta T}{\Delta t}$}

\begin{equation}
    \frac{\partial c_w}{\partial \frac{\Delta T}{\Delta t}} = - \frac{U^2}{R m_w \frac{\Delta T}{\Delta t}^2}
\end{equation}

\subsection{Fehlerfortpflanzung $c_w$}

\begin{itemize}
    \item $\bar{U} = \left( 0.013 \pm 0.006 \right) \left[\frac{K}{s}\right]$
    \item $\bar{\frac{\Delta T}{\Delta t}} = \left( 3.65 \pm 0.04 \right) \left[ V  \right]$
    \item $R = 1.7 \left[ \Omega \right]$
    \item $m_w = 132.1 \left[ g \right]$
\end{itemize}

\begin{equation}
    \sigma_{c_w} = \sqrt{ \left(\frac{\partial c_w}{\partial U}\right)^2 \sigma_U^2 + \left(\frac{\partial c_w}{\partial \frac{\Delta T}{\Delta t}} \right)^2 \sigma_{\frac{\Delta T}{\Delta t}}^2} 
\end{equation}

\begin{equation}
    \sigma_{c_w} = 1.9196 \left[\frac{J}{g K}\right] \approx 2 \left[\frac{J}{g K}\right]
\end{equation}

\centerline{\fbox{\colorbox{fh_green}{$c_{\text{water}} = \left( 5 \pm 2   \right) \left[ \frac{J}{g K}\right]$}}}


\section{Fehlerfortpflanzung: Herleitung $c_{\text{solid body}}$ auf Basis des ersten Thermodynamischen Gesetzes}

\begin{equation}
    \Delta Q_{\text{water}} \approx  \Delta Q_{\text{solid}}
\end{equation}

\begin{equation}
    m_w c_w (T_{\text{Eq}} - T_w) = m_{\text{solid}} c_{\text{solid}} (T_{\text{solid}} - T_{\text{Eq}})
\end{equation}

\subsection{Partielle Ableitung nach $c_w$}

\begin{equation}
    \frac{\partial c_{\text{solid}}}{\partial c_w} = \frac{m_w (T_{\text{Eq}} - T_w)}{m_{\text{solid}} (T_{\text{solid}} - T_{\text{Eq}})
}
\end{equation}

\subsection{Fehlerfortpflanzung $c_{solid}$}

\begin{itemize}
    \item $m_w = 120.5 [g]$
    \item $m_{solid} = 128.8 [g]$
    \item $T_{Eq} = 12.7 [^{\circ} C]$
    \item $T_w = 13.9 [^{\circ} C]$
    \item $T_{solid} = 13.9 [^{\circ} C]$
\end{itemize}

\begin{equation}
    \sigma_{c_{solid}} = \sqrt{ \left(\frac{\partial c_{solid}}{\partial c_w}\right)^2 \sigma_{solid}^2}
\end{equation}

\begin{equation}
    \sigma_{c_{solid}} \approx 0.12 \left[\frac{J}{g K}\right]
\end{equation}

\centerline{\fbox{\colorbox{fh_green}{$c_{solid} = \left(0.25 \pm 0.12\right) \left[\frac{J}{g K}\right]$}}}


%%%%%%%%%%%%%%%%%%%%%%%%%%%%%%%%%%% Inhalt (Ende)
\end{document}
%%%%%%%%%%%%%%%%%%%%%%%%%%%%%%%%%%% Dokument (Ende)
